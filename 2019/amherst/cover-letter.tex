\documentclass{letter}

\usepackage[margin=1in]{geometry}

\date{}
\signature{Jason K. Moore}
\address{2420 37th St \\ Sacramento, CA USA 95817}

\begin{document}

\begin{letter}{
  Department of Kinesiology \\
  School of Public Health and Health Sciences \\
  University of Massachusetts Amherst \\
  30 Eastman Lane \\
  Amherst, MA 01003
}

\opening{Dear Search Committee:}

I am writing to apply for a ``Assistant Professor - Biomechanics'' in the
Kinesiology Department. I am currently an Assistant Professor of Teaching in
the Mechanical and Aerospace Engineering Department at the University of
California, Davis. My position is an academic senate tenure-track position
primarily focused on advancing undergraduate engineering education. I spend
approximately 70\% of my time on teaching and learning, 20\% on professional
achievements and activities, and 10\% on academic and public service. I have
been in this position for over four years and my accomplishments over those
years align mostly with this position's teaching focus but I have also
maintained a research program centered on human mobility in transportation,
sports, and assistive technologies. Prior to this position, I was a
postdoctoral researcher the field of human motion and control with applications
in powered lower limb prostheses and before that a graduate student with
research in bicycle-rider dynamics, control, and handling in a sport
biomechanics lab. I believe that I am highly qualified for your department's
advertised position.

My current research trajectory is focused on developing human-machine
synergistic controllers for powered exoskeletons, powered prostheses, and
personal mobility vehicles, particularly single track vehicles. These assistive
devices will play a significant role in how the abled and disabled get around
in the future. Biomechanics and the study of human motion through data driven
control identification play a central role in my research methods and I would
bring specific expertise in computational modeling and optimal control of
musculoskeletal systems and experimental biomechanics. I am a applied
biomechanicist and leverage theory, experimentation, and computation equally
well in my work.

I have a strong record of obtaining competitive funding even though it is not a
requirement for my current position. As a graduate student, I initiated and
co-authored an awarded US National Science Foundation grant to study the
control of bicycles which was unique in both from the student initiative and
making the case to fund an oft-harder to fund topic. As a student, I was also
awarded a very selective Fulbright Scholarship to the Netherlands, where I
spent a productive year in the bicycle dynamics lab. Most recently, I was co-PI
on a successful \$5M grant from the US Department of Education, as well as
several internal grants for education activities over the last four years.

My extensive teaching record in dynamics and design includes courses in the
topics of mechanical design, mechanical vibrations, system dynamics, multibody
dynamics, and vehicle dynamics at the undergraduate (BSc) and graduate (MSc,
PhD) levels. My graduate multibody dynamics course has long been a core course
in our biomechanics training at UC Davis. I have mentored close to 500 BSc and
MSc students in over 100 engineering projects in the last four years that were
all partnered with research, non-profit, and industry organizations. I would be
interested in teaching and developing undergraduate courses in applied
biomechanics, experimental methods and instrumentation in biomechanics, applied
optimal control of human movement, and biorobotics; all of which I have
extensive research experience in.

The best fit for me is at a university has a department who is mission focused,
works together forwarding this common mission, and values continuous change all
while deeply valuing engineering education through teaching and research
mentorship. I have included my research plan that centers around human mobility
and a teaching statement that outlines my pedagogical practices.

Thank you for your consideration.

\closing{Sincerely,}

\end{letter}
\end{document}
