\documentclass{article}

\usepackage[margin=1in]{geometry}
\usepackage[colorlinks=true]{hyperref}

\title{Teaching Statement}
\author{Jason K. Moore}
\date{}

\begin{document}

\maketitle

My interest in mechanical engineering began during my childhood where I spent
many hours making, tinkering, and discovering how the built environment
functioned. These hands-on experiences allowed me to develop a desire to
discover the connection to mathematics, physics, and science in order to
improve my projects. Once that realization occurred, any barriers I had towards
learning the engineering fundamentals were lifted. I hope to bring similar
kinds of experiences and motivation to the students I teach.

\section*{Teaching Philosophy}
%
Traditional engineering pedagogy focuses heavily on lecture-based theoretical
courses paired with few hands-on laboratory courses. The aspects of creativity,
team work, realistic constraints, and an infinite possibility of problem
solutions are too often absent. These aspects are needed to develop the
practical side of engineering that students will require in their future
careers. Often, only the students who participate in extracurricular projects,
internships, or research positions make the connection from theory to practice
early on in the curriculum. Thus, the traditional structure may not be the best
way to maximize student understanding of engineering principles and instill the
agile problem solving methods they will need in the future.

Although there are variety of motivational reasons students pursue engineering,
I believe introducing the applied engineering practices early in education will
build passion and interest. Once this door is open, it is much easier to weave
in the skills the students need to become stronger engineers in their future
careers. I would like to see a curriculum that mimics the actual practice of
engineering through an iterative pattern of posing realistic problems followed
by a search for the necessary fundamentals, culminating in the application of
new knowledge to arrive at a solution. Like my childhood experiences, this
model provides interesting realistic problems that allow students to discover
the engineering fundamentals as opposed to presenting all of the fundamentals
before the interesting problems arrive.

Engineering students are capable of creating and solving problems when they
enter college. We should enrich their entire experience (especially in the first
year!) with challenges from real-world problems that leave them begging for the
knowledge and tools that they typically have to slog through during their first
years of school. Richard Miller, President of Olin College, often draws an
analogy between engineering students and violin students: ``Can you imagine not
playing the violin until your fourth year of study? Violinists start making
sounds with their instrument the first day of lessons.'' An engineering
curriculum could allow our students to draw the engineering bow across the
strings the minute they step into the classroom. For these early project-based
courses to be effective, however, the latest pedagogical developments must be
utilized to maximize learning potential.

\section*{Practical Classroom Examples}
%
In my courses, I try to provide students with open-ended problems that lead
into larger projects instead of problems designed for rote learning and
traditional exams. This approach more closely mimics the practice of
engineering. I combine this approach with rubric based assessments that set the
bar for mastery for improving student outcomes and effective assessment. I
attempt to have a good mixture of group and individual work, leaning more
heavily toward the former so students are prepared for the needs of industry. I
also have been working to orient my classrooms towards active learning. My best
example is the utilization of ``computational thinking'' that makes use of live
coding in class. I have setup a
\href{https://jupyter.libretexts.org}{JupyterHub server} that students log into
via laptops, tablets, and phones during class that provides an interactive
engineering computational environment. This allows access to my interactive
textbook that students use as a reading guide while I provide examples paired
with short computing exercises to periodically assess learning. I have
developed a related workshop for other practitioners with my colleague Allen
Downey from Olin College of Engineering which has been successful. I use exam
reflections surveys to provide self-guided preparation for subsequent exams.
Lastly, another very important method that I make use of is rapid in-class
assessment; at every break, each student provides me with anonymous quick
feedback: one line comments that share what they didn't understand and what was
effective.  This allows me to adjust my teaching after the break based on the
feedback. I tie this in with collected feedback before, during, and after the
course to have data to back my teaching decisions.

All of these methods are backed by evidence from education research. To keep
up-to-date on topics like these, I follow the education research literature,
especially the summary literature aimed at practicing educators and attend
``teach the teachers'' style workshops as much as possible. I have worked
closely with the UCD Center for Educational Effectiveness and the UCD
Engineering Education Learning Community these past four years to improve
student learning in my courses and for my department.

\section*{Prior Experience and Future Interests}
%
A teacher is often at their best when they know their material well. I spent
most of my graduate school years in the UCD MAE department and now four years
on the faculty making me intimately familiar with their undergraduate and
graduate curriculum. I have taught a number of the available courses as a
teaching assistant, lecturer, and professor. At the undergraduate level I have
strong experience with the dynamics and controls courses along with mechanics
and machine design curriculum. I have taught
``\href{http://www.moorepants.info/jkm/courses/eng4}{Engineering Graphics and
Design}'',
``Manufacturing Processes'',
``\href{http://moorepants.github.io/eng122}{Introduction to Mechanical
Vibrations}``,
''\href{http://moorepants.github.io/eme150a}{Mechanical Design}``,
''\href{https://moorepants.github.io/eme185}{Mechanical Systems Design Project}``,
''\href{https://moorepants.github.io/eme134}{Vehicle Stability}``,
and ''\href{https://moorepants.github.io/eme171}{Analysis, Simulation and
Design of Mechatronic Systems}``.
I have also taught ''\href{https://moorepants.github.io/mae223}{Multibody
Dynamics}`` at the graduate level. I believe my
prior experience makes me quite versatile and able to teach a broad variety of
courses.

In evidence based practices and innovations in the classroom. Some highlights
from the last four years are:

\begin{itemize}
  \setlength\itemsep{0.1em}
  \item developed a design competition and exchange program with Meijo
    University (Nagoya, Japan) on the cultural influences of robot and machine
    design
  \item flipped a mechanical vibrations class by utilizing ``computational
    thinking'' and project oriented learning with a custom designed interactive
    textbook and deployment through a JupyterHub server
  \item created a design studio classroom space that facilities active learning
    for our design courses
  \item created extensive rubric based assessment for written and oral
    communication in the capstone design course
  \item created a set of twenty Jupyter notebooks on multibody dynamics for
    in-class use and accompanying publicly available videos
  \item developed a transit bus bicycle rack design project which included
    reverse engineering, concept generation, and lightweight prototyping
  \item solicitation and mentoring of over 90 industry, government, and
    non-profit supported design projects spanning the mechanical engineering
    discipline
  \item co-awarded a \$5M Department of Education grant to create interactive
    OER engineering textbooks as part of the LibreTexts project
  \item co-wrote a book on teaching with Jupyter
  \item co-developed a workshop designed to teach STEM educators effective
    methods to teach with computation
\end{itemize}

There are at least four undergraduate courses that I would like to co-develop
in the future that are influenced by my research endeavors: 1) a first year
problem solving with data, simulation, and engineering computation, 2) an upper
level applied robotics and controls course centered around hands on work with
robotic vehicles, 3) an upper level elective focusing on project based
assistive device design, and 4) an introduction to computational optimization.

At the graduate level, I am also well prepared to teach many courses in
dynamics, control theory, biomechanics, optimization, software engineering, and
vehicle dynamics. I would like to continue to teach advanced multibody dynamics
from computational perspective. Additional ideas include developing a course
focusing on the design, simulation, and optimization of legged biomechatronics
that aligned closely with my post doctoral research. Students will learn about
neuromuscular modeling, mammalian gait, and get exposed to the latest tools in
the field (OpenSim, Biomechancal ToolKit, ROS/Gazebo, IPOPT, etc). Lastly, I
would very much an experimental biomechanics oriented course would also nicely
complement the computational oriented one to prepare students for applied work
in the field.

My course topic strengths are not entirely based on UCD MAE offerings. I have
spent time at Delft University of Technology, Old Dominion University,
Cleveland State University, Stanford University, and with the Software
Carpentry non-profit where I have gleaned both new course ideas and
methodologies to provide stronger connections to industry. I have experience in
teaching computational methods for data science. I have given numerous
workshops and tutorials to scientists and engineers on simulation,
optimization, and data analysis. I have been trained by the Software Carpentry
organization in pedagogical methods and teach two-day workshops around the
world to introduce scientists and engineers to the best practices and methods
in scientific computing. The mechanical engineer of the future will be
additionally tasked with data driven engineering. The engineering curriculum
will need to adapt to bring data science into many of the core courses for our
students to stay competitive in the job market, which I am ready to do.

\end{document}
