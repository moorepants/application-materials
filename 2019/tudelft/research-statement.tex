\documentclass{article}

\usepackage[a4paper,margin=25cm]{geometry}
\usepackage[colorlinks=true]{hyperref}

\title{Research Statement}
\date{}


\begin{document}

\maketitle

\section{Introduction}
%
My primary research efforts center on understanding and improving human
mobility by developing biomemetic controllers derived from measurements of
human motion during unassisted activities (walking, jumping, etc.) and
activities in which the human is operating a vehicle (bicycling, lower-limb
prostheses) or manipulating a tool (sporting equipment). I am most interested
in developing human-in-the-loop control systems that augment the human's
control experience for improved control, safe operation, and performance
enhancement.

The fundamental question that I am currently invested in is:

\begin{quote}
 Can dynamically human-similar or human-coupled machines and their controllers
 be designed to move as well as or better than a human controller would move,
 if provided neurally-limited driving control inputs?
\end{quote}

for powered assistive devices and humanoid
robots and on developing control augmentation for small manually controlled
vehicles. The ability humans have to ambulate and manipulate devices continues
to be the envy of engineers who desire to artificially mimic their motion.
Understanding and mimicking the intricacies of the mammalian neuromuscular
system have the potential to allow us to improve human life through assistive
machine and device design. However, contemporary robots and machines are still
limited in their ability to emulate the robust capabilities of mammalian
sensing and actuation.

I am interested in identifying practical controllers for devices, robots, and
vehicles which encode feed-forward and feedback control such that the combined
human/machine system has nearly identical motion to an able-bodied human. My
current focus is on improving balance while standing and walking with lower
limb prostheses and exoskeletons. To do this my research currently has three
primary foci:

\begin{enumerate}
  \item Identifying how humans balance and locomote through data intensive
    computational estimation, learning, and identification.
  \item Applying biomemetic control algorithms and design enhancements derived
    from identified controllers to assistive devices such as exoskeletons,
    powered prostheses, small vehicles, and humanoid robots.
  \item Developing next generation open and collaborative computational tools
    to back efforts in the first two items.
 \end{enumerate}

Single track vehicles offer a very appealing platform for studying mutlibody
dynamics and control (manual and automous).

\section{Current State of Single Track Vehicle Dynamics and Control}
%

\section{My Past Work in Human Motion and Control}
%
Much of my prior research has focused on the problem of control identification
in human balance where I have attempted to answer this question:

   Given the simultaneous measurements of the kinematics of human motion and
   optionally human/environment interface and internal system forces, what is
   the casual relationship from sensing to actuation in human motion?

My graduate work focused on understanding the control mechanisms humans use
while balancing on a bicycle. Because the bicycle is a dynamically complex
vehicle \cite{Astrom2005,Meijaard2007,Moore2007,Moore2008} that acts as an
intermediary between the human and the environment it is a powerful vehicle
platform for understanding balance.

I began by applying principal component analysis to a large collection of
motion capture data during steady state bicycling on a treadmill, which
identified dominant motion patterns and exposed subtle leg motions used for
balance at low speeds~\cite{Moore2009a,Moore2011c}. We further confirmed
this low speed behavior with video analysis of more natural bicycling behavior around a
city and on a treadmill~\cite{Kooijman2009}. Following those initial experiments, I
developed an extensively instrumented bicycle, capable of accurately measuring the full
dynamic state of the rider-vehicle system \cite{Moore2012,Moore2013} and
collected copious amounts of data during responses to lateral perturbations in
path tracking tasks. Using a manual control based theoretic controller we
developed~\cite{Hess2012} and data driven parameter estimation, I identified a set of
controllers that explained the dominant rider perturbation linear response
behavior~\cite{Moore2012}, which was then used to characterize a general controller
able to mimic human behavior for a broader set of control tasks. This was
expanded further with other theoretic controller structures for bicycling
\cite{Schwab2012a,Schwab2012,Schwab2013} and also applied to aircraft control
identification~\cite{Hess2013}.

The work on bicycle control identification lead into postdoctoral work focused
on developing controllers for lower extremity exoskeletons designed to assist
paraplegic individuals in walking. We partnered with Parker Hannifin Corp. and
targeted their Indego Exoskeleton. My goal was to provide natural gait and
unassisted balance for these devices, something that is still lacking today.
Utilizing an actuated treadmill coupled with full body kinematic tracking, I
collected large quantities of walking data from both normal walking and
longitudinally perturbed walking. I published the data as one of the first data
papers in the field \cite{Moore2015b} and demonstrated the effectiveness of the
treadmill belt perturbation method. I used this data with a direct gait cycle
gain scheduled feedback identification technique to identify possible closed
loop controllers \cite{Moore2013c,Moore2014a,Moore2014c}. This work led to the
development of an indirect identification technique based on parameter
estimation with direct collocation to enable simulated validation of the
controllers. Direct collocation gave us the computational speed to discretely
simulate hours of data. Starting with a simpler problem, I developed methods
with data derived from human perturbed standing data. The techniques led to
orders of magnitude of improvement in computation speed and control
identification strictly from kinematic data \cite{Moore2014e,Moore2015}.

Since moving into a teaching faculty position at UC Davis I have mentored and
led a number of sensing, instrumentation, and robotics projects that build on
the prior research with various local companies and undergraduate students. We
have developed an adaptive mouth-based control for an electric tricycle which
is ALS and quadriplegic friendly with Outrider USA and Disability Reports. This
past year my students developed a powered cable driven hand prostheses for
partial upper body paralysis with Ekso Bionics. With SRE Engineering we
developed a wireless boot for measuring ground reaction forces for horse
trotting in non-laboratory settings that I would like to apply to human
walking. I also mentored a group of students that developed a robot to tie a
shoe, one of the more complex tasks human hands perform. Lastly, I have
developed a desktop balancing robot that will be used to validate the indirect
identification methods for standing balance that I mentioned above.

With
undergraduate teams, I have published a paper on the design of safe ski
jumps~\cite{Moore2018} and have a paper under review in which we improved the
accuracy of rowing performance metrics using filtering techniques from
smartphone inertial measurement unit data~\cite{Cloud2019b}.


Some of my
current projects can be viewed on my lab website:
\url{http://mechmotum.github.io}.

All of my research relies heavily on open source computational data analysis
and simulation tools, much of which I have developed and published. Most
notably, I am a core developer of SymPy~\cite{SymPyDevelopmentTeam2006}, a
computer algebra system, and the maintainer of the classical mechanics
package~\cite{Gede2013}. Our 2017 paper \cite{Meurer2017} on the 11 year old
software has over 200 citations, along with thousands of users and hundreds of
contributors making it one of the most popular packages in the Scientific
Python ecosystem.  Additionally, I have developed a suite of bicycle dynamics
and control related software packages
\cite{Moore2010b,Moore2011d,Moore2011a,Moore2011e} and dynamics/biomechanics
packages \cite{Dembia2011,Moore2011,Moore2011b} \cite{Moore2013b}
\cite{Moore2017b} \cite{Moore2018a}. Recently I have published a package for
general purpose trajectory optimization and parameter estimation
\cite{Moore2018} and also for ski jump design \cite{Moore2018a}.

\bibliographystyle{ieee}
\bibliography{references}

\end{document}
