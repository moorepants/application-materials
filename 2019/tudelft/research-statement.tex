\documentclass{article}

\usepackage[a4paper,margin=25mm]{geometry}
\usepackage[colorlinks=true,citecolor=blue]{hyperref}
\usepackage[T1]{fontenc}  % for the Polish accents in bibliography

\title{Research Statement}
\author{Jason K. Moore}
\date{}

\begin{document}

\maketitle

\section*{Introduction}
%
My primary research efforts have centered on understanding and improving human
mobility by developing biomemetic controllers derived from data collected of
human motion during unassisted activities (e.g. standing, walking) and
activities in which the human is operating a vehicle (e.g. bicycling,
skateboarding) or manipulating a tool (e.g. leg prostheses, sporting
equipment). I am most interested in developing human control compatible machine
designs and human-in-the-loop control systems that augment a person's
experience for improved control, safe operation, and performance enhancement.
The fundamental question that I am currently focused on is:
%
\begin{quote}
 Can dynamically human-similar or human-coupled machines and their controllers
 be designed to move as well as or better than a human controller would move,
 if provided neurally-limited driving control inputs?
\end{quote}

To answer this question my research currently has three primary foci:
%
\begin{enumerate}
  \item Identifying how humans balance and locomote through data intensive
    computational estimation, learning, and identification.
  \item Applying biomemetic control algorithms and design enhancements derived
    from identified controllers to assistive devices such as bicycles,
    exoskeletons, powered prostheses, personal mobility vehicles, and humanoid
    robots.
  \item Developing next generation open and collaborative computational tools
    to back efforts in the first two items.
 \end{enumerate}

My prior research combines knowledge and methods derived from multibody
dynamics, sports biomechanics, human motion and control, transportation, and
computational engineering. I consider myself an applied computational and
experimental dynamics and control engineer; at home equally at the whiteboard,
computer, and lab. Much of my work utilizes single track vehicles as a platform
for studying multiibody dynamics and manual control.

\section*{My Past Work in Human Motion and Control}
%
Much of my prior research has focused on the problem of control identification
in human balance where I have attempted to answer this question:

\begin{quote}
  Given the simultaneous measurements of the kinematics of human motion and
  optionally human/environment interface and internal system forces, what is
  the casual relationship from sensing to actuation in human motion?
\end{quote}

% TODO : Yumi says this is too detailed and I show focus on the major
% achievements in this work.

My graduate work focused on understanding the control mechanisms humans use
while balancing on a bicycle. Because the bicycle is a dynamically complex
vehicle~\cite{Astrom2005,Meijaard2007,Moore2007,Moore2008} that acts as an
intermediary between the human and the environment, it is an ideal vehicle
platform for understanding balance and manual control.

My early work focused on applying principal component analysis to a large
collection of motion capture data during steady state bicycling on a treadmill,
which identified dominant motion patterns and exposed subtle leg motions used
for balance at low speeds~\cite{Moore2009a,Moore2011c}. We further confirmed
this low speed behavior with video analysis of more natural bicycling behavior
around a city and on a treadmill~\cite{Kooijman2009}. This work has since been
widely cited in the motion studies literature.

Following those initial experiments, I designed and fabricted an unique
instrumented bicycle, capable of accurately measuring the full dynamic state of
the rider-vehicle system~\cite{Moore2012,Moore2013}, including the most
accurate steer torque measurements to date. With this instrument, I collected
copious amounts of data during responses to lateral perturbations in path
tracking tasks. Then, starting with our theorectic manual control
model~\cite{Hess2012}, I applied data driven parameter estimation techniques to
arrive at set of controllers that explained the rider's dominant linear
response behavior to the perturbations~\cite{Moore2012}. I showed that this
control model was able to mimic human behavior for a broader set of control
tasks as well be used to estimate the handling qualities of different bicycle
designs. This work also supported other theoretic controller structures for
bicycling~\cite{Schwab2012a,Schwab2012,Schwab2013} and was also applied to
aircraft control identification~\cite{Hess2013}.

The work on bicycle control identification lead into postdoctoral work focused
on developing controllers for lower extremity exoskeletons designed to assist
paraplegic individuals in walking. We partnered with Parker Hannifin Corp. and
targeted their Indego Exoskeleton. My research was led by the goal to provide
natural gait and unassisted balance for these devices, something that is still
lacking today. Utilizing an actuated treadmill coupled with full body kinematic
tracking, I collected large quantities of walking data from both normal walking
and longitudinally perturbed walking. I published the data as one of the first
data papers in the field~\cite{Moore2015b} and demonstrated the effectiveness
of the treadmill belt perturbation method. The data has been used in a variety
of other control studies~\cite{Rohani2017,Thatte2018,Vimal2019}, as well as
inspiring more data papers~\cite{Santos2016,Fukuchi2017,Fukuchi2018,Santuz2018}.

I used the perturbed walking data with a direct gait cycle gain scheduled
feedback identification technique to identify possible closed loop
controllers~\cite{Moore2013c,Moore2014a,Moore2014c}. The technique proved to be
very computationally efficient and the identified controllers exhibited
repeatable feedback patterns in the gait cycle of across the walkers.

Difficulties in utility of the prior work, led to the development of an
indirect identification technique based on parameter estimation with direct
collocation to enable simulated validation of the
controllers~\cite{Moore2014e,Moore2015}.  We demonstrated the computational
efficiency and ability to accurately identify control from kinematic data in
simulated perturbed human standing. I further codified this technique in a
general purpose software library that efficiently constructs the optimization
or identification problem from a high level problem
description~\cite{Moore2014,Moore2018}.

\section*{My Current Work in Human Motion and Control}

Since moving into a teaching focused position at UC Davis, I have mentored and
led a number of sensing, instrumentation, software, and robotics projects with
various local companies and undergraduate students that build on my prior
research. We have developed an adaptive mouth-based control for an electric
tricycle which is quadriplegic friendly with Outrider USA and Disability
Reports and developed a powered cable driven hand prostheses for partial upper
body paralysis with Ekso Bionics. With SRE Engineering we developed a wireless
boot for measuring ground reaction forces for horse trotting in non-laboratory
settings that I would like to develop for human walking. I also mentored a
group that developed a robot to tie a shoe~\cite{Choi2018}, one of the more
complex tasks human hands perform.


I have further developed our theoretic single track vehicle handling quality
metric proposed in \cite{Hess2012} and utlized it in discovering a variety of
bicycle designs with theorectically optimal
handling~\cite{Moore2016,Moore2019a}. To validate this theory, we have
developed experimental methods with preliminarly results indicating a likely
relationship in the handling quality metric and rider subjective
ratings~\cite{Kresie}. And most recently, we have begun constructing the
optimal bicycles~\cite{Gilboa2019a} and evaluating them in terms of handling.

I have also led two projects in sports engineering in the past couple of years:
1) development of a web application for designing and analyzing ski jumps for
improving jumper safety~\cite{Moore2018a,Cloud2019a} and 2) the implementation
of real-time algorithms which improve estimation of competitive rowing
performance metrics from smartphone inertial measurement unit
data~\cite{Cloud2019b}.

All of my research relies heavily on open source computational data analysis
and simulation tools, many  of which I have developed and published. Most
notably, I am a core developer of the computer aided algebra system,
SymPy~\cite{SymPyDevelopmentTeam2006}, and the lead maintainer of the associated
classical mechanics package~\cite{Gede2013}. Our 2017 paper \cite{Meurer2017}
on the 11 year old software has over 250 citations, along with thousands of
users and hundreds of contributors making it one of the most popular packages
in the Scientific Python ecosystem. Additionally, I have developed a suite of
bicycle dynamics and control software
packages~\cite{Moore2010b,Moore2011d,Moore2011a,Moore2011e}, general purpose
dynamics and control packages~\cite{Moore2014,Moore2011}, and biomechanics
packages~\cite{Dembia2011,Moore2011,Moore2011b,Moore2013b}.

More about my research projects can be viewed on my lab website:
\url{http://mechmotum.github.io}.

\section*{My Bicycle Research Plans}
%
Over the last two decades, the use of bicycles for transportation has increased
in many countries that have a historically low mode share. This positive change
is the result of many factors, one of which is advances in bicycle related
technologies. In the near future, bicycles will be connected into the smart
city, we will see far greater use of electric bicycles, bicycles will be used
for cargo purposes in urban centers, and both older and younger riders will
grow in numbers. Yet, with the increase in ridership, bicyclists will remain
vulnerable road users and more political and technological solutions will be
needed to increase the safety. As a mechanical engineer, I plan to play an
integral role in growing bicycling in the world and making it safer through
technological advances.

The lab I envision will support developments in all types of bicycle related
technologies. Collaborations with other TU Delft researchers will help us make
advances in tire material properties, evaluation of electric bicycle
transmissions, power augmentation control in pedal assist electric bicycles,
ergonomics, performance improvements, etc.

As a professor focusing on bicycle dynamics and control, I will have several
parallel and intertwined avenues of research that will support this broader
social vision:
%
\begin{itemize}
  \setlength\itemsep{0.2em}
  \item Data driven vehicle and biomechanical modeling of the bicycle-rider
    system
  \item Data driven manual control characterization and identification in
    bicycle balance and navigation
  \item Design and augmented/autonomous control for improvements in bicycling
    safety and handling
  \item Development of the next generation of bicycle designs for improvements
    to transportation
  \item Application of vehicle dynamics to influence transportation
    infrastructure design
  \item Performance and safety improvements in the sport of cycling
\end{itemize}

I will initiate these research avenues by focusing on several low hanging fruit
among the current state of the art, my prior work, and the TU Delft Bicycle and
Control Laboratory's prior work. The following paragraphs detail the initial
projects I have in mind.

The most popular open loop bicycle vehicle dynamics model fails to make
accurate predictions of the vehicle's motion resulting from prescribed input
forces acting on the handlebars~\cite{Moore2013a}. Using data collected during
closed-loop riding (via robotic and/or human control) I want to identify and
codify the missing first principles. I also plan to further work in developing
a simple rider biomechanical model that captures the most important degrees of
freedom and actuation methods based on~\cite{Moore2011}. These two will advance
the state of the art vehicle-rider system models.

Identification of the controller in a closed-loop feedback and feed-forward
system, is not a simple task~\cite{vanderKooij2005}. I want to expand on the
robustness of my earlier efforts at this identification, by collecting data
during a broad set of events more rich in frequency content while collecting
additional physiological measures to capture signals related to the internal
sensing system of the human body. This identification will use both optimal
control and parameter identification theory as well as physics-informed machine
learning. This will result in general validated human control models for single
track vehicles.

To succeed at the above initiatives, I want to design and construct a new force
sensing treadmill for single track vehicle experimentation. Measuring the
ground reaction forces and moments at each individual tire while performing
maneuvers on a treadmill in a simulated environment will provide the next level
of fidelity in understanding the vehicle and rider dynamics. This treadmill
will be unique in the world and combined with simulation environments develop
at TU Delft will offer the most realistic and capable platform for single track
vehicle dynamics and control evaluation that will be attractive for many
industry interests.

Using both a robotic bicycle (currently on loan to TU Delft from UC Davis) and
TU Delft's steer-by-wire bicycle I plan to explore how vehicle and controller
design can effectively assist the rider's control, i.e. to take over when the
human's deficiencies are present and to maximize the performance in safety
critical maneuvering. These efforts will provide essential elements to having
autonomous single track vehicles in the future and safety enhancements that go
far beyond simple traction control for single track vehicles.

I want my research to have direct impact on transportation mode share and
safety for bicycling. I have preliminary work on using bicycle simulations for
evaluating bicycle transportation infrastructure and creating standards for
infrastructure. I call this ``physics-informed bicycle infrastructure
evaluation an design''. There is also an opportunity to apply bicycle balance
and control theory to the design of cargo bicycles and other non-traditional
bicycles.

Finally, I have long had a fantasy of being part of a team that could claim the
world human powered speed record. TU Delft has done so in the past and I intend
to help them do so again. I would like to apply my work in trajectory
optimization in both the speed challenge and other timed races. The are
additionally interesting opportunities to relieve the speed challenge rider
from balance control tasks so they can focus fully on power generation.

I am unfortunately at a disadvantage for acquiring funding in the Netherlands
and the EU due to my funding history all being within the United States system.
I will seek support from colleagues in the department to get up to speed
quickly on the various opportunities and leverage partnerships.

% Work with industry
% developin a theory on bicycle handling

\bibliographystyle{plain}
\bibliography{references}

\end{document}
