\documentclass{article}

\usepackage[a4paper,margin=25mm]{geometry}
\usepackage[colorlinks=true,citecolor=blue]{hyperref}
\usepackage[T1]{fontenc}  % for the Polish accents in bibliography

\title{Research Statement}
\author{Jason K. Moore}
\date{}

\begin{document}

\maketitle

\section*{Introduction}
%
My primary research efforts have centered on understanding and improving human
mobility by developing biomemetic controllers derived from data collected of
human motion during unassisted activities (e.g. standing, walking) and
activities in which the human is operating a vehicle (e.g. bicycling, skateboarding)
or manipulating a tool (e.g. leg prostheses, sporting equipment). I am most
interested in developing human control compatible machine designs and
human-in-the-loop control systems that augment the human's control experience
for improved control, safe operation, and performance enhancement. The
fundamental question that I am currently focused on is:
%
\begin{quote}
 Can dynamically human-similar or human-coupled machines and their controllers
 be designed to move as well as or better than a human controller would move,
 if provided neurally-limited driving control inputs?
\end{quote}

To answer this question my research currently has three primary foci:
%
\begin{enumerate}
  \item Identifying how humans balance and locomote through data intensive
    computational estimation, learning, and identification.
  \item Applying biomemetic control algorithms and design enhancements derived
    from identified controllers to assistive devices such as bicycles,
    exoskeletons, powered prostheses, personal mobility vehicles, and humanoid
    robots.
  \item Developing next generation open and collaborative computational tools
    to back efforts in the first two items.
 \end{enumerate}

My prior research combines knowledge and methods derived from multibody
dynamics, sports biomechanics, human motion and control, transportation, and
computational engineering. I consider myself an applied computaitonal and
experimental dynamics and control engineer. Much of my work utilizes single
track vehicles as a platform for studying mutlibody dynamics and manual
control.

\section*{My Past Work in Human Motion and Control}
%
Much of my prior research has focused on the problem of control identification
in human balance where I have attempted to answer this question:

\begin{quote}
  Given the simultaneous measurements of the kinematics of human motion and
  optionally human/environment interface and internal system forces, what is
  the casual relationship from sensing to actuation in human motion?
\end{quote}

My graduate work focused on understanding the control mechanisms humans use
while balancing on a bicycle. Because the bicycle is a dynamically complex
vehicle~\cite{Astrom2005,Meijaard2007,Moore2007,Moore2008} that acts as an
intermediary between the human and the environment it is an ideal vehicle
platform for understanding balance.

I began by applying principal component analysis to a large collection of
motion capture data during steady state bicycling on a treadmill, which
identified dominant motion patterns and exposed subtle leg motions used for
balance at low speeds~\cite{Moore2009a,Moore2011c}. We further confirmed this
low speed behavior with video analysis of more natural bicycling behavior
around a city and on a treadmill~\cite{Kooijman2009}. Following those initial
experiments, I developed an extensively instrumented bicycle, capable of
accurately measuring the full dynamic state of the rider-vehicle
system~\cite{Moore2012,Moore2013} and collected copious amounts of data during
responses to lateral perturbations in path tracking tasks. Using a manual
control based theoretic controller we developed~\cite{Hess2012} and data driven
parameter estimation, I identified a set of controllers that explained the
dominant rider perturbation linear response behavior~\cite{Moore2012}, which
was then used to characterize a general controller able to mimic human behavior
for a broader set of control tasks. This was expanded further with other
theoretic controller structures for
bicycling~\cite{Schwab2012a,Schwab2012,Schwab2013} and also applied to aircraft
control identification~\cite{Hess2013}.

The work on bicycle control identification lead into postdoctoral work focused
on developing controllers for lower extremity exoskeletons designed to assist
paraplegic individuals in walking. We partnered with Parker Hannifin Corp. and
targeted their Indego Exoskeleton. My goal was to provide natural gait and
unassisted balance for these devices, something that is still lacking today.
Utilizing an actuated treadmill coupled with full body kinematic tracking, I
collected large quantities of walking data from both normal walking and
longitudinally perturbed walking. I published the data as one of the first data
papers in the field~\cite{Moore2015b} and demonstrated the effectiveness of the
treadmill belt perturbation method. The data has been used in a variety of
other control studies by other researchers. I used this data with a direct gait
cycle gain scheduled feedback identification technique to identify possible
closed loop controllers~\cite{Moore2013c,Moore2014a,Moore2014c}. The identified
controllers provided insight into the feedback patterns in the gait cycle. This
work led to the development of an indirect identification technique based on
parameter estimation with direct collocation to enable simulated validation of
the controllers. Direct collocation gave us the computational speed to
discretely simulate hours of data. Starting with a simpler problem, I developed
methods with data derived from human perturbed standing data. The techniques
led to orders of magnitude of improvement in computation speed and control
identification strictly from kinematic data~\cite{Moore2014e,Moore2015}.

Since moving into a teaching faculty position at UC Davis I have mentored and
led a number of sensing, instrumentation, and robotics projects that build on
the prior research with various local companies and undergraduate students. We
have developed an adaptive mouth-based control for an electric tricycle which
is ALS and quadriplegic friendly with Outrider USA and Disability Reports. This
past year my students developed a powered cable driven hand prostheses for
partial upper body paralysis with Ekso Bionics. With SRE Engineering we
developed a wireless boot for measuring ground reaction forces for horse
trotting in non-laboratory settings that I would like to apply to human
walking. I also mentored a group of students that developed a robot to tie a
shoe, one of the more complex tasks human hands perform. Lastly, I have
developed a desktop balancing robot that will be used to validate the indirect
identification methods for standing balance that I mentioned above.

Over the last few years I have further developed our theorectic single track
vehicle handling quality metric~\cite{Hess2012} and formulate a bicycle
optimal design method that adapts CMA-ES to work with constraint equations to
discover a suite of ununual bicycle designs that have optimal lateral handling
qualities~\cite{Moore2016,Moore2019a}. Furthermore, we have been developig the
experimental methods to validate the handling
predictions~\cite{Kresie2017,Gilboa2019a}

With undergraduate teams, I have published a paper on the design of safe ski
jumps~\cite{Moore2018} and have a paper under review in which we improved the
accuracy of rowing performance metrics using filtering techniques from
smartphone inertial measurement unit data~\cite{Cloud2019b}.

All of my research relies heavily on open source computational data analysis
and simulation tools, much of which I have developed and published. Most
notably, I am a core developer of SymPy~\cite{SymPyDevelopmentTeam2006}, a
computer algebra system, and the maintainer of the classical mechanics
package~\cite{Gede2013}. Our 2017 paper \cite{Meurer2017} on the 11 year old
software has over 200 citations, along with thousands of users and hundreds of
contributors making it one of the most popular packages in the Scientific
Python ecosystem.  Additionally, I have developed a suite of bicycle dynamics
and control related software packages
\cite{Moore2010b,Moore2011d,Moore2011a,Moore2011e} and dynamics/biomechanics
packages \cite{Dembia2011,Moore2011,Moore2011b} \cite{Moore2013b}
\cite{Moore2017b} \cite{Moore2018a}. Recently I have published a package for
general purpose trajectory optimization and parameter estimation
\cite{Moore2018} and also for ski jump design \cite{Moore2018a}.

More about my research projects can be viewed on my lab website:
\url{http://mechmotum.github.io}.

\section*{My Bicycle Research Plans}
%
Over the last two decades, the use of bicycles for transportation has been
increasing in many countries that have historically low mode share. This
positive change is the result of many factors, one of which is advances in
technology. In the near future, bicycles will be connected into the smart city,
we will see far greater use of electric bicycles, bicycles will be used for
cargo purposes in urban centers, and both older and younger riders will grow in
numbers. Yet bicyclists will remain vulnerable road users and with the increase
in ridership and more political and technological solutions will be needed to
increase the safety. As a mechanical engineer, I plan to play an integral role
in growing bicycling in the world and making it safer through technological
advances.

As a professor focusing on bicycle dynamics and control, I will have several
parallel and intertwined avenues of research that will support this broader
vision:
%
\begin{itemize}
  \item Data driven vehicle and biomechanical modeling of the bicycle-rider
    system
  \item Data driven manual control characterization and identification in
    bicycle balance and navigation
  \item Augmented and autonomous control for improvements in bicycling safety
    and handling
  \item Development of the next generation of bicycle designs for improvements
    to transportation
  \item Application of vehicle simulation to influence transportation
    infrastructure design
  \item Performance and safety improvements in the sport of cycling
\end{itemize}

I will initiate these research avenues by focusing on several low hanging fruit
based on the current state of the art and a amalgamation of my prior work and
the TU Delft Bicycle and Control Laboratory's prior work. The following
paragraphs detail the initial projects I have in mind.

The most popular open loop bicycle vehicle dynamics model fails to make
accurate predictions of the vehicle's motion resulting from prescribed input
forces acting on the handlebars~\cite{Moore2013a}. Using data collected during
closed-loop riding (via robot and/or human) I want to work to discover the
first principles that are missing from the current model. I also plan to
further work in developing a simple rider biomechanical model that captures the
most important degrees of freedom and actuation methods based
on~\cite{Moore2011}.

Identification of the controller in a closed loop feedback and feedforwad
system, is not a straight forward task~\cite{vanderKooij2005}. I want to expand
on the robustness of my earlier efforts at this identification, by collecting
data during a broad set of events rich in frequency content while collecting
additional physiological measures to capture signals related to the internal
sensing system of the human body.

To do the above, I want to design and construct a new force sensing treadmill
for single track vehicle experimentation. Measuring the ground reaction forces
and moments at each individual tire while performing maneuvers on a treadmill
in a simulated environment will provide the next level of fidelity in
understanding the vehicle and rider dynamics. This treadmill will be unique in
the world and combined with simulation environments develop at TU Delft will
offer the most realistic and capable platform for single track vehicle dynamics
and control studies.

Using both a robotic bicycle (currently on loan to TU Delft from UC Davis) and
TU Delft's steer-by-wire bicycle I plan to explore how vehicle and controller
design can effectively assist the rider's control, i.e. to take over when the
human's deficiencies are present and to maximize the performance in safety
critical maneuvering. These efforts will provide essential elements to having
autonomous single track vehicles in the future and safety enhancements that go
far beyond simple traction control for single track vehicles.

I want my research to have direct impact on transportation mode share and
safety for bicycling. I have preliminary work on using bicycle simulations for
evaluating bicycle transportation infrastructure and creating standards for
infrastructure. I call this ``physics informed bicycle infrastructure
evaluation an design''. There is also an opportunity to apply bicycle balance
nd control theory to the design of cargo bicycles and other non-traditional
bicycles.

I have long had a fantasy of being part of a team that could clam the work
human powered speed record. TU Delft has done so in the past and I intend to
help them do so again. I would like to apply my work in trajectory optimization
in both the speed challenge and other timed races. The are additionally
interesting opportunities to relieve the speed challenge rider from balance
control tasks so they can focus fully on power generation.

The lab I envision will also support developments in all types of bicycle and
related technologies. Examples may be advances in tire material properties,
evaluation of gearing systems, power augmentation control in pedal assist
electric bicycles, ergonomics, etc.

I am unfortunately at a disadvantage for acquiring funding in the Netherlands
and the EU due to my funding history all being within the United States system.
I will seek support from colleagues in the department to get up to speed
quickly on the various opportunities and leverage partnerships.

% Work with industry
% consumer cycling and cargo bicycles, but also sports applications such as track cycling, road cycling and BMX
% developin a theory on bicycle handling

\bibliographystyle{plain}
\bibliography{references}

\end{document}
