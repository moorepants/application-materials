\documentclass{letter}

\usepackage[a4paper,margin=25mm]{geometry}

\date{}
\signature{Jason K. Moore}
\address{2420 37th St \\ Sacramento, CA USA 95817}

\begin{document}

\begin{letter}{
  Department of Biomechanical Engineering \\
  Delft University of Technology \\
  Delft, Netherlands
}

\opening{Dear Search Committee:}

I am writing to apply for the ``Assistant/Associate Professor (tenure track) in
Dynamics of Multibody systems'' in the Department of Biomechanical Engineering
at TU Delft. For the last 15 years, I have developed unique research expertise
in the dynamics and control of bicycles as well as other topics that align with
your department's research areas, making me a highly qualified candidate for
this position. I have authored or co-authored 21 journal and conference
articles on the subject of bicycles and my doctoral work focused almost
exclusively on the topic, resulting in a dissertation entitled ``Human Control
of a Bicycle'' which is well cited. I have published 8 other articles, many of
which were ancillary to the bicycle dynamics and control research. These works
and my active role in the associated academic communities make me one of the
handful of people in the world with the expertise needed to take the TU Delft
Bicycle Dynamics and Control lab into the future.

I am currently an Assistant Professor of Teaching in the Mechanical and
Aerospace Engineering Department at the University of California, Davis. My
position is an academic senate tenure-track position primarily focused on
advancing undergraduate engineering education. I spend approximately 70\% of my
time on teaching and learning, 20\% on professional achievements and
activities, and 10\% on academic and public service. I have been in this
position for over four years and my accomplishments over those years align
mostly with this position's teaching focus. I have also maintained a research
program centered on human mobility in transportation, sports, and assistive
technologies with much of the work advancing engineering in the dynamics and
control of bicycles. Prior to this position, I was a postdoctoral researcher in
the field of human motion and control with applications in powered lower limb
exoskeletons. Before that, I was a graduate student with research in bicycle
dynamics, control, and handling in a sport biomechanics lab. Through these
experiences, I have developed strong applied computational and experimental
vehicle dynamics and biomechanics experience.

My current research trajectory is focused on developing human-machine
synergistic controllers for powered exoskeletons, powered prostheses, and
personal mobility vehicles, particularly single track vehicles. These assistive
devices will play a significant role in how abled and disabled individuals get
around in the future.

I am well versed in the various funding opportunities in the USA and have a
strong record of obtaining competitive funding even though it is not a
requirement for my current position. As a graduate student, I initiated and
co-authored an awarded US National Science Foundation grant to study the
control of bicycles. This effort was unique given both the student initiative
and the fact that we were able to make the case for obtaining resources for a
research area that is often more challenging to fund. As a student, I was also
awarded a very competitive Fulbright Scholarship to the Netherlands, where I
spent a productive year in the bicycle dynamics lab at TU Delft. Most recently,
I was co-PI on a successful \$5 million grant from the US Department of
Education. Over the last four years I have also been awarded several internal
grants for education activities. I am looking forward to learning about the
funding systems that support faculty research in the Netherlands so that I can
continue my successful trajectory. At TU Delft, I plan to fund my future work
through grants and industry partnerships in the Netherlands, European Union,
and internationally.

My extensive teaching record in dynamics and design includes courses in the
topics of mechanical design, mechanical vibrations, system dynamics, multibody
dynamics, and vehicle dynamics at the undergraduate (BSc) and graduate (MSc,
PhD) levels. I make use of many education research backed practices in my
teaching and I am constantly improving the courses using these practices. I
have taught courses of up to 120 students and managed up to four teaching
assistants per class. I am also a leader in the use and promotion of
computational thinking for learning, recently co-authoring a book entitled
``Teaching and Learning with Jupyter'' which provides and introduction to the
related methods and tools. I look forward to expanding this for the large
courses at TU Delft.

Over the past four years, I have also mentored nearly 500 BSc and MSc students
in over 100 engineering projects. As these projects involved external project
sponsors, I developed relationships with a diverse set of research, non-profit,
and industry organizations. Several of these projects are also with
international groups in Kenya, Nicaragua, Cambodia, and Sweden. I am especially
proud of the mechanical design exchange program I have developed with Meijo
University in Japan. As my native language is English, I will be able to teach
MSc classes immediately at TU Delft, but I also have a beginner understanding
of Dutch and am eager to become fluent so I can teach BSc courses.

Lastly, I have a strong service record. I serve on my department's
undergraduate curriculum committee where we are modernizing our course
offerings. I serve on the scientific and planning committees of the Bicycle and
Motorcycle Dynamics conference series and recently hosted the International
Cycling Safety Conference at UC Davis for the first offering outside of Europe.
On the education front, I am a topic editor of the innovative Journal of Open
Source Education, now in its second year of publishing.

If at TU Delft, I see numerous opportunities for collaboration within the
university. It is exciting that so many authors of papers I have read are at
this institution. In the Biomechanical Engineering department, Prof.  van der
Helm and Prof. Harlaar's expertise in upper and lower limb biomechanics will
provide insight for bicycling biomechanics. Prof. Vallery's work in
biomechatronics and teaching dynamics can align nicely with this position's
efforts. I have interacted with Prof. Seth when he was at Stanford and his
foundational work and expertise in musculoskeletal simulation are a strong
interest of mine. I have studied and cited Prof. van der Kooij's control
identification work and would be excited to learn directly from him.  Dr.
Geijtenbeek's predictive simulation methods and tools are leading this research
area and  his connections to Motek Medical would be helpful for my experimental
work. I have co-authored a paper with Prof. Happee in the Department of
Transport and Planning and will seek collaborations related to vehicle
engineering. The Control and Simulation group has strong expertise in manual
control theory and identification that I have leveraged in the past. For
example, I have worked with Prof. van Paassen on the Python control software
and could expand that effort and Prof. Mulder's work in manual control has
benefited my applications to single track vehicles. Lastly, I am excited to
develop collaborations with Dutch bicycle companies being that the country
leads the world in innovative utility bicycle designs.

The best career fit for me is to be a faculty member at a university department
that is mission focused, works together forwarding this common mission, and
strives for continuous change all while deeply valuing engineering education
through teaching and research mentorship.


I have included my research plan that centers around human mobility and a
teaching statement that outlines my pedagogical practices in addition to how I
think I would fit into your department.

Thank you for your consideration.

\closing{Sincerely,}

\end{letter}
\end{document}

%- [x] internationally oriented
%- [x] successes in competitve funding
%- [x] international partnerships: academic and industry
%- [x] strengthen their existing reserach lines and add expertise
%- produce PhDs and [x] pubs
%- [x] teach bs and ms dynamics courses
%- [x] supervise bs, Mmc projects and [ ] supervise phds
%- [x] innovate teaching for large classses
%- [x]] service to scientific communities society, univseristy
%- [x] work with non profits
%- promote tu delft
%- [x] experience running lab doing experiments using modern equip
%- [x] exp teaching BSc and msc
%- like to create and innovate in university context
%- good communicator: enthusiam empahty, comminit passion
%- you can inspiree and motivate, coach and delegate, deal with diversity
%- [x] english
%- [x] willingnesss to leanr dutch

