\documentclass{article}

\usepackage[margin=1in]{geometry}
\usepackage[colorlinks=true,citecolor=blue]{hyperref}
\usepackage[T1]{fontenc}  % for the Polish accents in bibliography

\title{Research Statement}
\author{Jason K. Moore}
\date{}

\begin{document}

\maketitle

\section*{Introduction}
%
My primary research efforts have centered on understanding and improving human
mobility by developing biomemetic controllers derived from data collected on
human motion during unassisted activities (e.g. standing, walking) and
activities in which the human is operating a vehicle (e.g. bicycling,
skateboarding) or interacting with a machine or implement (e.g. leg prostheses,
sporting equipment). I am most interested in developing human control
compatible machine designs and human-in-the-loop control systems that augment a
person's experience for improved control, safe operation, and performance
enhancement. The fundamental question that I am currently focused on is:
%
\begin{quote}
 Can dynamically human-similar or human-coupled machines and their controllers
 be designed to move as well as or better than a human controller would move,
 if provided neurally-limited control inputs?
\end{quote}

To answer this question, my current research has three primary foci:
%
\begin{enumerate}
  \setlength\itemsep{0.1em}
  \item Identifying how humans balance and locomote through data intensive
    computational estimation, learning, and identification.
  \item Applying biomemetic control algorithms and design enhancements derived
    from identified controllers to assistive devices such as bicycles,
    exoskeletons, powered prostheses, personal mobility vehicles, and humanoid
    robots.
  \item Developing next generation open and collaborative computational tools
    to back efforts in the first two items.
 \end{enumerate}

My prior research combines knowledge and methods derived from multibody
dynamics, sports biomechanics, human motion and control, transportation, and
computational engineering. I consider myself an applied computational and
experimental dynamics and control engineer; equally comfortable at the
whiteboard, on the computer, and in the lab. Much of my work utilizes single
track vehicles as a platform for studying multibody dynamics and manual
control.

\section*{My Past Work in Human Motion and Control}
%
Much of my prior research has focused on the problem of control identification
in human balance where I have attempted to answer this question:

\begin{quote}
  Given the simultaneous measurements of the kinematics of human motion and
  optionally human/environment interface and internal system forces, what is
  the casual relationship from sensing to actuation in human motion?
\end{quote}

My graduate work focused on understanding the control mechanisms humans use
while balancing on a bicycle. Because the bicycle is a dynamically complex
vehicle~\cite{Astrom2005,Meijaard2007,Moore2007,Moore2008} that acts as an
intermediary between the human and the environment, it is an ideal vehicle
platform for understanding balance and manual control.

My early graduate work focused on applying principal component analysis to a
large collection of motion capture data during steady state bicycling on a
treadmill, which identified dominant motion patterns and exposed subtle leg
motions used for balance at low speeds~\cite{Moore2009a,Moore2011c}. We further
confirmed this low speed behavior with video analysis of more natural bicycling
behavior around a city and on a treadmill~\cite{Kooijman2009}. This work has
since been widely cited in the motion studies literature.

Following those initial experiments, I designed and fabricated a uniquely
instrumented bicycle, capable of accurately measuring the full dynamic state of
the rider-vehicle system~\cite{Moore2012,Moore2013}, including the most
accurate steer torque measurements to date. With this instrument, I collected
copious amounts of data during responses to lateral perturbations in path
tracking tasks. Then, starting with our theorectic manual control
model~\cite{Hess2012}, I applied data driven parameter estimation techniques to
arrive at a set of controllers that explained the rider's dominant linear
response behavior to the perturbations~\cite{Moore2012}. I showed that this
control model was able to mimic human behavior for a broader set of control
tasks and could be used to estimate the handling qualities of different bicycle
designs. This work also supported other theoretic controller structures for
bicycling~\cite{Schwab2012a,Schwab2012,Schwab2013} and was also applied to
aircraft control identification~\cite{Hess2013}.

The work on bicycle control identification lead into postdoctoral work focused
on developing controllers for lower extremity exoskeletons designed to assist
paraplegic individuals in walking. We partnered with Parker Hannifin Corp. and
targeted their Indego Exoskeleton. My research was led by the goal to provide
natural gait and unassisted balance for these devices, something that is still
lacking today. Utilizing an actuated treadmill coupled with full body kinematic
tracking, I collected large quantities of walking data from both normal walking
and longitudinally perturbed walking. I published the data as one of the first
data papers in the field~\cite{Moore2015b} and demonstrated the effectiveness
of the treadmill belt perturbation method. The data has been used in a variety
of other control studies (e.g.~\cite{Rohani2017,Thatte2018,Vimal2019}), as well
as inspiring more data
papers~\cite{Santos2016,Fukuchi2017,Fukuchi2018,Santuz2018}.

I used the perturbed walking data with a direct gait cycle gain scheduled
feedback identification technique to identify possible closed loop
controllers~\cite{Moore2013c,Moore2014a,Moore2014c}. The technique proved to be
very computationally efficient and the identified controllers exhibited
repeatable feedback patterns in the gait cycle across the walkers.

Difficulties in utility of the prior work led to the development of an indirect
identification technique based on parameter estimation with direct collocation
to enable simulated validation of the controllers~\cite{Moore2014e,Moore2015}.
We demonstrated the computational efficiency and ability to accurately identify
control from kinematic data in simulated perturbed human standing. I further
codified this technique in a general purpose software library that efficiently
constructs the optimization or identification problem from a high level problem
description~\cite{Moore2014,Moore2018}.

\section*{My Current Work in Human Motion and Control}
%
Since moving into a teaching focused position at UC Davis, I have mentored and
led a number of sensing, instrumentation, software, and robotics projects with
various local companies and undergraduate students that build on my prior
research. We have developed an adaptive mouth-based control for an electric
tricycle which is quadriplegic friendly with Outrider USA and Disability
Reports and developed a powered cable driven hand prostheses for partial upper
body paralysis with Ekso Bionics. With SRE Engineering we developed a wireless
boot for measuring ground reaction forces for horse trotting in non-laboratory
settings that I would like to develop for human walking. I also mentored a
group that developed a robot to tie a shoe~\cite{Choi2018}, one of the more
complex tasks human hands perform.

I have further developed our single track vehicle handling quality metric
proposed in \cite{Hess2012} and utilized it in discovering a variety of bicycle
designs with theoretically optimal handling~\cite{Moore2016,Moore2019a}. To
validate this theory, we have developed experimental methods with preliminary
results indicating a likely relationship in the handling quality metric and
rider subjective ratings~\cite{Kresie2017}. And most recently, we have begun
constructing the optimal bicycles~\cite{Gilboa2019a} and evaluating them in
terms of their handling.

I have also led two projects in sports engineering in the past couple of years:
1)~development of a web application for designing and analyzing ski jumps for
improving jumper safety~\cite{Moore2018a,Cloud2019a} and 2)~the implementation
of real-time algorithms that improve estimation of competitive rowing
performance metrics from smartphone inertial measurement unit
data~\cite{Cloud2019b}.

All of my research relies heavily on open source computational data analysis
and simulation tools, many of which I have developed and published. Most
notably, I am a core developer of the computer aided algebra system,
SymPy~\cite{SymPyDevelopmentTeam2006}, and the lead maintainer of the
associated classical mechanics package~\cite{Gede2013}. Our 2017 paper
\cite{Meurer2017} on the now 13 year old software has over 250 citations, along
with thousands of users and hundreds of contributors making it one of the most
popular packages in the scientific Python ecosystem. Additionally, I have
developed a suite of bicycle dynamics and control software
packages~\cite{Moore2010b,Moore2011d,Moore2011a,Moore2011e}, general purpose
dynamics and control packages~\cite{Moore2014,Moore2011}, and biomechanics
packages~\cite{Dembia2011,Moore2011,Moore2011b,Moore2013b}.

More about my research projects can be viewed on my lab website:
\url{http://mechmotum.github.io}.

\section*{My Research Plans%
  \label{my-research-plans}%
}

As a new professor I will play an integral role in your department's vision for
rehabilitation and movement science. I plan to lead a laboratory that will
provide computational and experimental biomechanics expertise alongside
humanoid robot and assistive device design. I hope to contribute to the
department with a modernized biomechatronics research and teaching focus.

With more than a decade in Northern California, I have a wide network of
partners to bring this vision to life that span orthotics companies to Bay Area
biomechanic and robotics companies. My network also spans beyond the region to
the state, national, and international collaborations. I plan to grow my
collaborations with regional companies and labs (e.g. Toyota Research
Institute, Motion Analysis, Ekso Bionics, Inscitech, Open Robotics, Stanford's
Neuromuscular Biomechanics Lab) along with my expanded collaborators (e.g.
Cleveland State's Human Motion and Control Laboratory, Cornell's Biorobotics
Lab, TU Delft's Biomechanics Department, and Meijo University). I also look
forward to developing more cross disciplinary research partners, many which
have begun with the 90+ capstone design projects I have mentored.

I will continue to participate in a number of academic communities that I am
currently involved with. The lab will target conferences such as Dynamic
Walking and ROSCon along with the American and International Societies of
Biomechanics (including the ISB Technical Simulation group). On the software
side, we will continue to present at SciPy, PyData, and PyCon for open source
computation.

The lab I am planning will be able to 1) collect motion data from humans and
robots in mobility related activities both in the lab and in natural
environments, 2) apply cutting edge learning, estimation, and identification
methods to characterize human control, 3) build and test controllers in
humanoid robots and assistive devices, and 4) contribute to and develop the
next generation of open source biomechatronic related software.

My initial project plans are multifold and will build from my prior work. I
will start recruiting students for 1) applying parameter identification using
direct collocation to perturbed walking data to discover a gain scheduled
closed loop control, 2) development of a scaled balancing robot that simulates
perturbed human balancing, 3) accelerating lower body neuromuscular forward
dynamics simulations through implicit dynamics and optimized code generation
and common sub-expression evaluation across CPU/GPU cores, and 4) development
of a low-fidelity lower limb exoskeleton for controller testing.

Given the opportunity, I have the skills, network, and vision to succeed as a
professor of kinesiology \& health.

\bibliographystyle{plain}
\bibliography{references}

\end{document}
