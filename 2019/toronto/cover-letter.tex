\documentclass{letter}

\usepackage[margin=1in]{geometry}

\date{}
\signature{Jason K. Moore}
\address{2420 37th St \\ Sacramento, CA USA 95817}

\begin{document}

\begin{letter}{
  Faculty of Kinesiology \& Physical Education \\
  University of Toronto \\
  Toronto, Ontario, Canada
}

\opening{Dear Search Committee:}

I am writing to apply for the ``Associate Professor in Biomechanics or Human
Movement Control'' in the Faculty of Kinesiology \& Physical Education at the
University of Toronto. For the last 15 years, I have developed unique research
expertise in human motion and control as well as other topics that align with
your University's research areas, making me a highly qualified candidate for
this position. I have authored or co-authored 29 journal and conference
articles on the subject of human motion and control. These works, my active
role in the associated academic communities, and my future plans that align
with your needs in biomechanics, kinesiology, and human systems integration
make me an ideal candidate.

My current research trajectory is focused on developing human-machine
synergistic controllers for powered exoskeletons, powered prostheses, and
personal mobility vehicles, particularly single track vehicles. These assistive
devices will play a significant role in how abled and disabled individuals get
around in the future.

I am currently an Assistant Professor of Teaching in the Mechanical and
Aerospace Engineering Department at the University of California, Davis. My
position is an academic senate tenure-track position primarily focused on
advancing undergraduate engineering education. I spend approximately 70\% of my
time on teaching and learning, 20\% on professional achievements and
activities, and 10\% on academic and public service. I have been in this
position for over four years and my accomplishments over those years align
mostly with this position's teaching focus. I have also maintained a research
program centered on human mobility in transportation, sports, and assistive
technologies with much of the work advancing engineering in human balance
control. Prior to this position, I was a postdoctoral researcher with Prof. Ton
van den Bogert in the field of human motion and control where I developed data
driven control algorithms for powered lower limb exoskeletons. Before that, I
was a graduate student with research in bicycle dynamics, control, and handling
in Mont Hubbard's Sport Biomechanics Lab. I have also performed biomechanics
research with Prof. Arend Schwab at TU Delft as a Fulbright Scholar and with
Scott Delp at Stanford as a National Center Simulation in Rehabilitation
Research Fellow. Through these experiences, I have developed strong applied
computational and experimental biomechanics experience.

My research experience can contribute to all four of your faculty's foci. My
most recent work about optimizing performance of competitive rowers and my work
developing methods in trajectory optimization for optimal biomechanical motion
will tie into the focus \emph{optimizing performance}. I have led and
contributed to numerous sports projects from recent skiing and rowing to tennis
racket dynamics, fatigue in jumping with Nike, and bicycling; all of which tie
into \emph{healthy sport and active living} and \emph{global citizenry in sport
and exercise}. Lastly, my primary research objectives of improving human
balance through assistive control will support \emph{physical activity
interactions with healthcare}. My well rounded strengths in the field will
offer ample collaboration opportunities with your faculty.

I am well versed in the various funding opportunities in the USA and have a
strong record of obtaining competitive funding even though it is not a
requirement for my current position. As a graduate student, I initiated and
co-authored an awarded US National Science Foundation grant to study the
balance and control of bicycles. This effort was unique given both the student
initiative and the fact that we were able to make the case for obtaining
resources for a research area that is often more challenging to fund. As a
student, I was also awarded a very competitive Fulbright Scholarship to the
Netherlands, where I spent a productive year with the biomechanics faculty at
TU Delft. Most recently, I was co-PI on a successful \$5 million grant from the
US Department of Education. Over the last four years I have also been awarded
several internal grants for education activities. I am excited to learn about
the funding opportunities that Canada offers and will be able to adjust
quickly.

My extensive teaching record in dynamics and design includes courses in the
topics of mechanical design, mechanical vibrations, system dynamics, multibody
dynamics, and vehicle dynamics at the undergraduate (BSc) and graduate (MSc,
PhD) levels. My graduate multibody dynamics course is a core course for our
biomechanics students. I make use of many education research backed practices
in my teaching and I am constantly improving the courses using these practices.
I have taught courses of up to 150 students and managed up to four teaching
assistants per class. I am also a leader in the use and promotion of
computational thinking for learning, recently co-authoring a book entitled
``Teaching and Learning with Jupyter'' which provides and introduction to the
related methods and tools. I will be able to teach theoretical and experimental
biomechanics, musculoskeletal modeling and simulation, optimal control of
biomechanical movement, sports biomechanics, and design of biomedical products
at the University of Toronto.

Over the past four years, I have also mentored nearly 500 BSc and MSc students
in over 100 engineering projects. As these projects involved external project
sponsors, I developed relationships with a diverse set of research, non-profit,
and industry organizations. Several of these projects are also with
international groups in Kenya, Nicaragua, Cambodia, and Sweden. I am especially
proud of the mechanical design exchange program I have developed with Meijo
University in Japan. This network and experience will be valuable for your
department.

Lastly, I have a strong service record. I serve on my department's
undergraduate curriculum committee where we are modernizing our course
offerings and working closely with students to help them succeed. I serve on
the scientific and planning committees of the Bicycle and Motorcycle Dynamics
conference series and recently hosted the International Cycling Safety
Conference at UC Davis for the first offering outside of Europe. On the
education front, I am a topic editor of the innovative Journal of Open Source
Education, now in its second year of publishing.

The best career fit for me is to be a faculty member at a university department
that is mission focused, works together forwarding this common mission, and
strives for continuous change all while deeply valuing engineering education
through teaching and research mentorship.

I have included my research plan that centers around human mobility and a
teaching statement that outlines my pedagogical practices in addition to how I
think I would fit into your department.

Thank you for your consideration.

\closing{Sincerely,}

\end{letter}
\end{document}
