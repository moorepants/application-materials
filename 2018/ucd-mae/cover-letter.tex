\documentclass{letter}

\usepackage[margin=1in]{geometry}
\usepackage[colorlinks=true]{hyperref}

\date{}
\signature{Jason K. Moore}
\address{Jason K. Moore\\2420 37th St \\ Sacramento, CA USA 95817}

\begin{document}

\begin{letter}{
  Professor Rida T. Farouki \\
  Search Committee Chair \\
  Multibody \& System Dynamics -- Machine Design \& Robotics \\
  Department of Mechanical and Aerospace Engineering \\
  University of California, Davis \\
  One Shields Avenue \\
  Davis, CA 95616
}

\opening{Dear Search Committee:}

I am writing to apply for both positions in the call ``Two Assistant Professor
Positions in the areas of 1) Multibody \& System Dynamics and 2) Machine Design
\& Robotics'' in the Mechanical and Aerospace Engineering Department at the
University of California, Davis. I am currently an Assistant Professor of
Teaching Mechanical and Aerospace Engineering (LPSOE) in the same department.
My current position is focused on advancing engineering education in our
department which I believe I am excelling at if my merit reviews and other
external indicators are correct. I have been in this position for four academic
years and my accomplishments over that time align with this position's heavy
teaching and educational focus. I have spent 11 of the last 14 years at UC
Davis and I have a deep connection to the both the academic and regional
communities. It is a cherished place for me and I have vested interest in my
career path here.

My primary goal at this stage in my academic career is to move into a typical
professorship where I can continue to excel at teaching and education but also
have the resources and support I need to focus on and further my research
program in human mobility started during my graduate and postdoctoral
positions. My current position has systematic hurdles that prevent me from
bringing my research potential to full bear even if I excel at my primary
educational duties. For example, many incoming graduate students want to work
with me but the lack of lab space is an immediate deterrent. This unfortunately
leaves me with a deep longing for the research I love. Regardless, I have been
able to maintain a modest research output with a mostly undergraduate team on
top of excelling at my education related activities.

My prior very active research record and what I have maintained in the last
four years are reflected in the 10 peer reviewed journal publications, 16 peer
reviewed conference proceedings, 2 draft books, 12 software packages, 30+
invited and conference talks, and over \$5M I have helped bring to the campus.
My collective works have been cited over 550 times and I have an h-index of
12~\footnote{Google Scholar:
\href{http://tinyurl.com/jkm-gscholar}{tinyurl.com/jkm-gscholar}}, which is
favorably comparable to a number of faculty already in the department. I am
also a leader in my respective dynamics, biomechanics, safety, and computing
research communities by chairing conference sessions, organizing conferences,
editing journals, and acting on scientific committees.  With the resources
these positions offer I will be able to launch from this already strong
platform and develop as a key research contributor to the department and
campus.

The UCD MAE department is at a clear turning point in its life due to the
numerous retiring faculty in the past decade. These faculty defined what the
world knows us for. It is certainly time to bring in many new people with new
ideas but we also must steward our legacy strengths into the promising and
dynamic future of the next generation of mechanical and aerospace engineering.
Both advertised positions highlight this implicit desire. With this in mind, I
am in a very unique position due to my strong institutional knowledge of our
department and familiarity with the curriculum. I have collaborated with
numerous researchers at UC Davis and have a strong regional research and
engineering network tied to over a hundred projects I have been involved with.
I have taught eight of our undergraduate courses and taken thirteen of our
courses while a graduate student. All of these courses are in the areas of
multibody \& system dynamics, machine design \& robotics, and controls. My
current course offerings are aligned precisely with the advertised positions
and are a natural fit.

But, I also bring in new ideas. I have spent 8 years at four other
institutions. My postdoctoral training was from two of the leading
biomechanicists in the world and my Fulbright year offered experience at one of
the world's top technical universities. These experiences have provided with me
with a diverse academic background and stimulus for innovation to help
transform our curriculum and research activities for the next 30 years.
Additionally, many of the retiring and retired professors that these two
positions are meant to fill roles for would vouch for my likelihood of
excellent stewardship and the ability to infuse modern research and engineering
themes into the future curricula. For example, I plan to revive our lost
strength in biomechanics but born anew in the form of biomechatronics to lead
us into the future.

My current research trajectory is centered around developing data-driven
human-machine synergistic controllers for powered exoskeletons, powered
prostheses, personal mobility vehicles (electric bicycles, scooters,
wheelchairs, skateboards), and other assistive devices. These assistive
vehicles and devices will play a significant role in how everyone, particularly
the abled and disabled, young and old, get around both indoors and outdoors in
the future. I will weave these device and vehicle technology focused efforts
into the efforts to transform our country's transportation system so that we
can have a clean, multi-modal, energy efficient, accessible modes to move
people from place to place. For example, one aspect that this position could
enable is an in depth exploration of the role smart self-balancing personal
mobility vehicles and devices can play in enhancing the active lifestyle of
elderly individuals. Also, electric bike share system companies have a growing
interest in control augmented vehicles for safety and the ability to
autonomously return bicycles to charging stations. This is an avenue that I
plan to pursue aggressively with help from our Institute for Transportation
studies new BicyclingPlus research collaborative.

I have included my research plan that centers around human mobility and how it
fits here at UC Davis and a teaching statement that outlines my pedagogical
practices and how I plan to contribute to new curricula if hired.

Thank you for your consideration.

\closing{Sincerely,}

\end{letter}
\end{document}
