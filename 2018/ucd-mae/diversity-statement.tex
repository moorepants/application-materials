\documentclass[titlepage]{article}

\usepackage[margin=1in]{geometry}
\usepackage[colorlinks=true]{hyperref}

\title{Statement of Contributions to Diversity}
\date{}

\begin{document}

\maketitle

In terms of understanding how diversity plays an important role in the growth
of our societies and the lives of the people in them, I have grown a great deal
from my pre-adult years where I grew up in the still very racially segregated
and homophobic small town southern United States. There I was born into a place
of privilege due to my skin color, gender, and socioeconomic status. I
recognize now how much those factors have played into my upward movement
throughout my life and how my hard work is not the only thing that got me where
I am today. My view of how the world works and what diversity means has had
many positive transformational changes over the years but I would like to
highlight one of the more powerful experiences I have had that has affected how
I behave and make decisions when I am in a teaching role.

The mechanical engineering field unfortunately needs much improvement in
attracting women and having a racially diverse student population, not to
mention other diversification needs. Engineering has a long history of bias
towards the status quo that educational leaders today are trying to unravel and
set straight. Oddities like the dominance of women in early computer science
and the rapid decline of their participation are both functions of our
intentional and unintentional decisions and behavior. Many of these ingrained
societal influences are beyond our control as teachers, but I will be dedicated
to improving this situation with the power that I have. I have been very
fortunate to be involved with several radical communities in terms of
diversification over the years and believe those experiences will help improve
diversity at UCD if I am hired.

I spent eight years running and volunteer teaching at a do-it-yourself bicycle
repair shop which also traditionally suffers from gender and race imbalance but
we challenged this head on. Using techniques I gleaned from numerous trainings
on creating environments for people of all backgrounds, I was involved with
implementing the latest advice from cultural studies to make the shop as
inclusive as possible. I plan to extend the practices and knowledge from that
experience to the teaching atmosphere at the UCD MAE department to help us
create the most welcoming and inclusive engineering department in the country.

My tenure as an instructor at the bicycle shop taught me many things but the
most significant takeaways were not to take diversity, inclusion, and
marginalization lightly and definitely not to dismiss things that I have not
experienced or do not understand. I also learned to listen to people who think
about these things a lot and let their guidance influence my behavior and
decisions. I now have a strong support group to turn to for advice in difficult
situations. Overall, I have a better awareness and now recognize much more
quickly when situations are not ``right'' and am willing to stand up for
diverse students needs and know where the best avenues for help are.

There are also more specific examples of practices that I have picked up and
utilize when teaching. I set ground rules in classroom interaction early on
that helps ensure equitable time for students to speak so that traditional
dominators cannot control classroom time. Training has provided better ability
to recognize these patterns and facilitate classroom discussions so that they
are inclusive. I have worked to ensure groupings in team projects are diverse
to create stronger teams. I have also been involved with and observed
developing specific times and spaces for marginalized communities, like women,
LGBT, etc. This lets similar groups of learners learn on their own terms
instead of those of the dominant majority. I also work to develop classroom and
lab ground rules, such as codes of conduct and safe space standards, that are
in place for all students to see and be aware of, whether posted on a sign in
the lab or on the classroom website. I explicitly discuss these ground rules in
class and even let the students collaboratively develop these agreements so
they are invested in abiding by them. I have also learned how a classroom or
lab atmosphere can be exclusive to many groups simply because of things like
decor, lighting, politically incorrect jokes, and general attitudes and
behaviors of superiors.  I will be working with various groups to reduce and
eliminate these factors in my courses and labs. I have fought for more
equitable proposal evaluations when in that capacity. Furthermore, I work to
ensure that there are anonymous feedback avenues for students and work to place
students in need with the appropriate campus groups for support, letting them
and the support groups help inform what I need to do to make the classroom
accommodating.

Finally, I will be an ally for minority groups in the engineering college and
give what support I can to help them strengthen and grow and will be a strong
proponent of diversification of our selected students, staff, and faculty. My
ethnicity, gender, and socioeconomic status puts me in the ``typical engineer''
bucket in terms of diversification but my experiences in life working with and
for marginalized people, from disabled wheelchair fabricators in Zambia to
disenfranchised homeless at the DIY bicycle shop, has instilled the empathy and
understanding deep inside that will play an important role in changing
engineering stereotypes for the next generation.

\end{document}
