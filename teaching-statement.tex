\documentclass{article}

\usepackage[margin=1in]{geometry}
\usepackage[colorlinks=true]{hyperref}

\title{Teaching Statement}
\date{}

\begin{document}

\maketitle

My interest in mechanical engineering began in my childhood where I spent many
hours making, tinkering, and discovering how the built environment worked.
These hands-on experiences allowed me to develop a desire to discover the
connection to mathematics and physics in order to improve my projects. Once
that realization occurred, any barriers I had towards learning the engineering
fundamentals were lifted. I hope to bring similar kinds of experiences and
motivation to the students I teach.

\section*{Teaching Philosophy}
%
Traditional engineering pedagogy typically focuses heavily on lecture based
theoretical courses paired with some hands-on laboratory courses. The aspects of
creativity, team collaboration, and an infinite possibility of problem
solutions are all too often absent. As a result, undergraduate students are
often left to their own devices outside of the lecture and lab to discover the
skill development and practical side of engineering that they will require in
their future careers. Cookbook instructions for lab experiments or canned
problems with a single solution are virtually never present in the real world.
Often, only the students who participate in extracurricular projects,
internships, or research positions make the connection from theory to practice
early on in the curriculum. Thus, the traditional structure may not be the best
way to maximize student understanding of core engineering principles and
instill the agile problem solving methods they will need in the future.

Although there are variety of motivational reasons students pursue engineering,
I believe introducing the actual practices of engineering early in education
will build passion and interest. Once this door is open, it is much easier to
thread in the skills the students need to become stronger engineers in their
future careers. I would like to see engineering curriculum that mimics the
practice of engineering through an iterative pattern of posing realistic
problems followed by a search for the necessary fundamentals and finally
culminating in the application of the new knowledge to arrive at a solution to
the problem. Not unlike my childhood experience, this model allows interesting
realistic problems to lead students to the engineering fundamentals as opposed
to trying to present all of the fundamentals before the interesting problems
arrive.

Engineering students are capable of creating and solving problems when they
enter college. We should enrich their entire experience (especially the first
year!) with challenges from real-world problems that leave them begging for the
knowledge and tools that they typically have to slog through during their first
years of school. Richard Miller, President of Olin College, often draws an
analogy between engineering students and violin students: ``Can you imagine not
playing the violin until your fourth year of study? Violinists start making
sounds with their instrument the first day of lessons.'' Our curriculum could
allow our students to draw the engineering bow across the strings the minute
they step into the classroom. For these early project-based courses to be
effective, however, the latest pedagogical developments must be utilized to
maximize learning potential.

\section*{Practical Classroom Examples}
%
In my courses, I provide students with small-scale projects either based on
portions of the projects and problems I have had to solve in practice or from
Resources like the Directory of Projects for First-Year Engineering
Students~\footnote{\url{http://www.discovery-press.com/discovery-press/studyengr/projects.asp}}.
I typically focus on team success and ensure the problems are open-ended to
more closely mimic the practice of engineering. I choose classroom teaching
methods that have evidence-based research backing their tenets. For example,
research shows that peer learning is highly likely to improve both
understanding and retention when intertwined with short lectures. There are
already prominent examples on the UCD campus where these methods are making a
big difference. A recent New York Times
article~\footnote{\url{http://www.nytimes.com/2014/12/27/us/college-science-classes-failure-rates-soar-go-back-to-drawing-board.html}}
highlights the UCD Chemistry department's success with peer learning. Teachers
that are excited about new teaching methods can increase students'
understanding and retention of material more than those that stick with tired
curriculum. Another method that I make use of is rapid in-class assessment; at
every break, each student writes a positive or negative feedback/comment about
the last lecture portion on sticky note, passes it in, and I adjust my teaching
after the break based on the feedback. I tie this in with electronically
collected feedback before, during, and after the course to have data to back my
teaching decisions. To keep up-to-date on topics like these, I follow the
education research literature, especially the summary literature aimed at
practicing educators and plan to attend more ``teach the teachers'' style
workshops in the future.

\section*{Prior Experience and Future Interests}
%
A teacher is often at their best when they know their material well. I was fortunate
to have spent most of my graduate school years in the UCD MAE department and
now three years on the faculty making me intimately familiar with the
undergraduate and graduate curriculum. I have taught a number of the available
courses as a teaching assistant, lecturer, and professor. At the undergraduate
level I have strong experience with mechanics and machine design courses along
with the dynamics and controls curriculum. I have taught ENG 004, EME 050, ENG
122, EME150A, EME 150B, EME185A, EME185B, MAE 223, MAE 297 and will add EME 134
and EME 171 to the list this coming year. In my capacity as a teaching oriented
professor I have worked on introducing evidence based practices and innovations
in the classroom. Some highlights from the last three years are:

\begin{itemize}
  \item developed a design competition and exchange program with Meijo
    University (Nagoya, Japan) on the cultural influences of robot and machine
    design
  \item flipped a mechanical vibrations class by utilizing ``computational
    thinking'' and project oriented learning with a custom designed interactive
    textbook and deployment through a JupyterHub server
  \item created a design studio classroom space that facilities active learning
    in design courses
  \item created extensive rubric based assessment for written and oral
    communication in the capstone design course
  \item created a set of twenty jupyter notebooks on multibody dynamics for
    in-class use and accompanying publicly available videos
  \item developed a transit bus bicycle rack design project which included
    reverse engineering, concept generation, and lightweight prototyping
  \item solicitation and mentoring of over 70 industry, government, and
    non-profit supported design projects
\end{itemize}

I have three new undergraduate courses that I would like to develop in the
future that bring in modern topics to our curriculum and are influenced
For our undergraduate offereings I would like to 

I am also well prepared to teach many of the currently offered MAE graduate
level courses in dynamics, control theory, biomechanics, and vehicle dynamics.
I have taught Multibody Dynamics once and would like to continue to develop
that into a modern version of the course as it will play an essential role for
my graduate students.

I am interested in developing more modern courses in biomechanics, optimal
control, and the latest in multibody dynamics (e.g. O(N) methods).

modern experiemntal biomechanics focusing on collection of large data and
analytics

This puts me in a good position to maintain and grow many of the department's
core courses.

However, my course topic strengths are not entirely based on
current UCD MAE offerings.

I have spent time at Delft University of Technology, Old Dominion University,
Cleveland State University, Stanford University, and with the Software
Carpentry non-profit where I have gleaned both new course ideas and
methodologies to provide stronger connections to industry. For example, I have
experience in teaching computational methods for data science. I have given
numerous workshops and tutorials to scientists and engineers on simulation,
optimization, data analysis, etc. I have been trained by the Software Carpentry
organization in pedagogical methods and teach two-day workshops around the
world to introduce scientists and engineers to the best practices and methods
in scientific computing. The mechanical engineer of the future will be
additionally tasked with data driven engineering. The engineering curriculum
will need to adapt to bring data science into many of the core courses for our
students to stay competitive in the job market.

Finally, I would like to offer teaching workshops in collaboration with the
engineering college and education schools for teaching assistants (TAs) in our
department to improve TA instruction. Some of the topics I want to work on are
co-instruction, collaborative teaching material development, technology in the
classroom, and distilling the latest findings in engineering education to the
faculty, lecturers, and teaching assistants. Furthermore, I hope to serve on
college and campus wide committees that have education improvement topics and
participate in pedagogy trainings at other institutions so that I can bring
back new ideas to our department. For example, UC Davis has created the ``I am
STEM'' program dedicated to improving our university's STEM education efforts.
As a lecturer I will be working closely with this program and others to enhance
our offerings in the MAE department. The people that I meet at these campus
groups will be ideal collaborators for pursuing educational grant proposals
such as those under the NSF's Undergraduate Education (DUE) program. This
houses directorates like ``Improving Undergraduate STEM Education'' and more
interdisciplinary ones such as ``Transforming Undergraduate Education in
Science, Technology, Engineering and Mathematics'' that could help empower the
work done in the MAE department.

\end{document}
